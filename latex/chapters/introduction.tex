Epigenetics \cite{holliday2006epigenetics} studies the heritable traits that cannot be explained by changes in the DNA sequence. Examples of epigenetic mechanisms include DNA methylation and histone modification. These mechanisms adjust the expression level of genes \cite{jaenisch2003epigenetic}, which allows organisms to dynamically adapt to changes in the environment.

Disruption of gene expression levels is related to the development of various diseases \cite{egger2004epigenetics}. For example, the epigenetic deactivation of certain tumor suppressor genes commonly leads to the development of cancer \cite{esteller2008epigenetics}. The expression levels of certain genes can therefore be used as additional tools in early diagnostics of cancer, as prognosis factors and as predictors of response to treatment.

Good understanding of the relationship between DNA methylation and gene expression is important for both cancer prevention and epigenetic treatment. We have used the gene methylation and expression level data discussed in \cite{cancer2012comprehensive} to explore this relationship. One of the goals in this project is to produce a map that shows the methylation of which genes affects the expression levels of each gene.

\pagebreak

This is the introduction where you should introduce your work.  In
general the thing to aim for here is to describe a little bit of the
context for your work --- why did you do it (motivation), what was the
hoped-for outcome (aims) --- as well as trying to give a brief
overview of what you actually did.

It's often useful to bring forward some ``highlights'' into 
this chapter (e.g.\ some particularly compelling results, or 
a particularly interesting finding). 

It's also traditional to give an outline of the rest of the
document, although without care this can appear formulaic 
and tedious. Your call. 