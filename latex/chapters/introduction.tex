Epigenetics \cite{holliday2006epigenetics} studies the heritable traits that cannot be explained by changes in the DNA sequence. Examples of epigenetic mechanisms include DNA methylation and histone modification. These mechanisms adjust the expression level of genes \cite{jaenisch2003epigenetic}, which allows organisms to dynamically adapt to changes in the environment.

Disruption of gene expression levels is related to the development of various diseases \cite{egger2004epigenetics}. For example, the epigenetic deactivation of certain tumor suppressor genes commonly leads to the development of cancer \cite{esteller2008epigenetics}. The expression levels of certain genes can therefore be used as additional tools in early diagnostics of cancer, as prognosis factors and as predictors of response to treatment.

Good understanding of the relationship between DNA methylation and gene expression is important for both cancer prevention and epigenetic disease treatment. We have used the gene methylation and expression level data discussed in \cite{cancer2012comprehensive} to explore this relationship. One of the goals in this project is to produce a map that shows the methylation of which genes affects the expression levels of each gene.

Several methods \cite{tibshirani1996regression,zou2005regularization,li2008network,li2010variable,pan2010incorporating,luo2012two,kim2013network} have been implemented and considered for use with real data. The hyperparameters for each method have been tuned with the use of synthetic datasets as suggested in \cite{li2008network}. This is done because ground truth remains unknown for the relationship between gene methylation and expression.

A novel method of hyperparameter tuning is developed as an alternative to the widely used method of minimizing the cross-validated mean squared test error. In our context we have a bundle of regression methods that operate on the same training data and share a common goal - to correctly identify the relationship between predictor and target variables. Instead of tuning the various regression methods independently, our approach performs cooperative hyperparameter tuning on all methods simultaneously. It uses an iterative algorithm to increase the similarity of estimated coefficients for the various methods by tuning their hyperparameters. For each method and iteration, the method's parameter grid neighborhood is searched for a set of parameters that maximizes the correlation between its estimated coefficients and the averaged estimates of all other methods for the previous iteration. When this process converges a set of parameters is defined for each method that maximizes the overall agreement across the whole set of methods.

The two parameter tuning approaches are implemented and compared. After selecting a set of parameters 

outline of structure  