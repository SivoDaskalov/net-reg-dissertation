\chapter{Summary and Conclusions}
Several methods for linear regression found in literature \cite{tibshirani1996regression,zou2005regularization,li2008network,li2010variable,pan2010incorporating,luo2012two,kim2013network}, some of which using prior knowledge in the form of a network of predictor relationships, have been studied and implemented. The approaches perform regression by minimizing different objective functions, but in the context of this research they share common input data.

Synthetic datasets have been generated following a procedure similar to that of \cite{li2008network}. They are designed to resemble real epigenetic datasets and also provide ground truth about the relationships between their predictors and the target variable.

A method for merging of the regression estimates in the context of multiple regression methods operating on common input data has been developed. The proposed composite voting regression uses the estimates of its underlying ensemble of regression methods to determine whether each of the predictors is related to the target variable. Each regression method "votes" for the importance of a predictor if its estimate of the predictor's coefficient is non-zero. The subset of selected predictors are those which achieve a fraction of votes above a certain threshold. Ordinary least squares estimation is then performed only using the set of selected variables.

%We have developed a novel method of hyperparameter tuning which performs simultaneous cooperative tuning on an ensemble of regression methods.

%A novel method of hyperparameter tuning is developed as an alternative to the widely used method of minimizing the cross-validated mean squared test error. In our context we have a bundle of regression methods that operate on the same training data and share a common goal - to correctly identify the relationship between predictor and target variables. Instead of tuning the various regression methods independently, our approach performs cooperative hyperparameter tuning on all methods simultaneously. It uses an iterative algorithm to increase the similarity of estimated coefficients for the various methods by tuning their hyperparameters. For each method and iteration, the method's parameter grid neighborhood is searched for a set of parameters that maximizes the correlation between its estimated coefficients and the averaged estimates of all other methods for the previous iteration. When this process converges a set of parameters is defined for each method that maximizes the overall agreement across the whole set of methods.

%Good understanding of the relationship between DNA methylation and gene expression is important for both cancer prevention and epigenetic disease treatment. We have used gene methylation and expression level data from breast cancer patients, discussed in \cite{cancer2012comprehensive}, to explore this relationship. One of the goals in this project is to produce a map that shows the methylation of which genes affects the expression levels of each gene.

%Several methods \cite{tibshirani1996regression,zou2005regularization,li2008network,li2010variable,pan2010incorporating,luo2012two,kim2013network} have been implemented and considered for use with real data. The hyperparameters for each method have been tuned with the use of synthetic datasets as suggested in \cite{li2008network}. This is done because ground truth remains unknown for the relationship between gene methylation and expression.


%The similarities between estimates of the various regression methods have been explored. We also look into the effect that a choice of hyperparameter tuning procedure has on these similarities.

%The following chapters discuss the theoretical and implementation details of each linear regression approach. Afterwards, we introduce the synthetic dataset generation process used in our simulations. We then define our estimate merging procedure and our proposed cooperative hyperparameter tuning approach. In the simulations carried out we compare the various regression approaches, as well as the traditional and our proposed cooperative parameter tuning approach. Analysis of similarities between the various regression methods is carried out. The final chapter defines the real dataset used for exploration of the relationship between gene methylation and expression, presents and discussed the obtained results.
