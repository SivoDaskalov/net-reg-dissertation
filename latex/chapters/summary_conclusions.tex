\chapter{Summary and Conclusions}
Several methods for linear regression found in literature \cite{tibshirani1996regression,zou2005regularization,li2008network,li2010variable,pan2010incorporating,luo2012two,kim2013network}, some of which using prior knowledge in the form of a network of predictor relationships, have been studied and implemented. The various approaches perform regression by minimizing different objective functions, but in the context of this research they share common input data.

Synthetic datasets have been generated following a procedure similar to that of \cite{li2008network}. They are designed to resemble real breast cancer epigenetic datasets while also providing ground truth about the relationships between their predictors and the target variable.

A method for merging of coefficient estimates in the context of multiple regression methods operating on common input data has been developed. The proposed composite voting regression uses the estimates of its underlying collection of regression methods to determine whether each of the predictors is related to the target variable. Each regression method "votes" for the importance of a predictor if its estimate of the predictor's coefficient is non-zero. The subset of selected predictors are those which achieve a fraction of votes greater than or equal to a certain threshold. Ordinary least squares estimation is then performed only using the set of selected variables.

We have developed a novel method of hyperparameter tuning which performs simultaneous cooperative tuning on an ensemble of regression methods. It uses an iterative process which aims to increase the similarity between coefficient estimates produced by the various methods. Our proposed parameter tuning approach introduces cooperation between otherwise completely independent regression methods. This is done to reduce their individual overfitting, as well as promote agreement between their coefficient estimates. The performance of the proposed orchestrated hyperparameter tuning method depends on multiple factors, including the choice of regression methods in the ensemble, their individual hyperparameter search grids and the initial starting points of the tuning process. Several possible modifications to the baseline orchestrated tuning approach are proposed and discussed. One direction of future work would be to implement, evaluate and compare them.

Various metrics, including prediction error and variable selection capabilities, have been used to evaluate and compare the different regression methods, as well as the two hyperparameter tuning approaches (traditional and orchestrated). The comparison is carried out with the use of the synthetic datasets. The orchestrated hyperparameter tuning achieved performance similar to that of the tradition tuning. Slight improvement or deterioration was observed across the various comparison metrics and regression methods.

Similarity between the coefficient estimates of the various regression methods is also evaluated through the cosine similarity measure. Use of the proposed orchestrated hyperparameter tuning approach resulted in increased similarity of estimates for each pair of regression methods. This indicates that the desired effect of improved cross-method consensus is achieved.

Gene methylation and expression data from breast cancer patients is used to explore how the expression level of each gene is affected by the methylation levels of related genes. Methylation probes in the promoter region are considered separately from those in the gene body. A subset of the regression methods is used to model the relationships between gene methylation and expression for both the gene body and promoter region datasets. The produced mappings are capable of estimating the vector of gene expression levels given a vector of gene methylation levels. Future work could include comparison and evaluation of similarity between the mappings derived from methylation data in the promoter region and the gene body.

Hyperparameter combinations for use on the real breast cancer data have been selected for each linear regression method based on the distributions of synthetic dataset tuning outcomes. The number of edges in the breast cancer gene network used is significantly higher than that of the network used with the synthetic datasets. This could be the reason why parameter combinations obtained from tuning on the synthetic datasets perform poorly on the real breast cancer datasets. Two solutions could be considered in future work to mitigate these differences. The first approach is to use a synthetic gene network which more closely resembles the real breast cancer gene network. Alternatively, a different method of building the breast cancer gene network could be considered. 
